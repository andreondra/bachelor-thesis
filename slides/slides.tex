\documentclass{beamer}
\usetheme{CVUTFIT}

\title[Emulátor konzole NES]{Emulátor konzole Nintendo~Entertainment~System}
\subtitle{Bakalářská práce}
\author[Ondřej Golasowski]{Ondřej Golasowski\texorpdfstring{\\}{}Vedoucí práce: Ing.~Stanislav Jeřábek}
\institute[FIT ČVUT]{Katedra číslicového návrhu, FIT ČVUT}
\date{\today}

\begin{document}
	
\begingroup
\setbeamertemplate{footline}{}
\begin{frame}[noframenumbering]
	\titlepage
\end{frame}
\endgroup

\section{Úvod}
\begin{frame}
	\frametitle{Představení problematiky}
	\begin{definice}[Emulátor]
		Emulátor je software, který umožňuje běh počítačových programů na jiné platformě, než pro kterou byly původně vytvořeny.
	\end{definice}
\end{frame}

\begin{frame}
	\frametitle{Cíle a~motivace}
	\begin{itemize}
		\item Využití emulátorů ve vzdělávání --- výuka počítačových architektur.
	\end{itemize}
\end{frame}

\section{Praktická část}
\begin{frame}
	\frametitle{Implementace}
	Text...
\end{frame}

\begin{frame}
	\frametitle{Shrnutí}
	Text...
\end{frame}

\appendix

\section{Otázky}
\begin{frame}
	\frametitle{Otázky oponenta}
	Ještě žádné nejsou. :(
\end{frame}
	
\end{document}