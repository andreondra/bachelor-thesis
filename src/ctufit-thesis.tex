%% This is the ctufit-thesis example file. It is used to produce theses
%% for submission to Czech Technical University, Faculty of Information Technology.
%%
%% Get the newest version from
%% https://gitlab.fit.cvut.cz/theses-templates/FITthesis-LaTeX
%%
%%
%% Copyright 2021, Eliska Sestakova and Ondrej Guth
%%
%% This work may be distributed and/or modified under the
%% conditions of the LaTeX Project Public Licenese, either version 1.3
%% of this license or (at your option) any later version.
%% The latest version of this license is in
%%  https://www.latex-project.org/lppl.txt
%% and version 1.3 or later is part of all distributions of LaTeX
%% version 2005/12/01 or later.
%%
%% This work has the LPPL maintenance status `maintained'.
%%
%% The current maintainer of this work is Ondrej Guth.
%% Contact ondrej.guth@fit.cvut.cz for bug reports.
%% Alternatively, submit bug reports into the tracker at
%% https://gitlab.fit.cvut.cz/theses-templates/FITthesis-LaTeX/issues
%%
%%

%%%%%%%%%%%%%%%%%%%%%%%%%%%%%%%%%%%%%%%%%
% CLASS OPTIONS
% language: czech/english/slovak
% thesis type: bachelor/master/dissertation
% colour: bw for black&white OR no option for default colour scheme
%%%%%%%%%%%%%%%%%%%%%%%%%%%%%%%%%%%%%%%%%
\documentclass[czech,bachelor,unicode]{ctufit-thesis}

%%%%%%%%%%%%%%%%%%%%%%%%%%%%%%%%%%
% FILL IN THIS INFORMATION
%%%%%%%%%%%%%%%%%%%%%%%%%%%%%%%%%%
\ctufittitle{Emulátor konzole Nintendo Entertainment System} % replace with the title of your thesis
\ctufitauthorfull{Ondřej Golasowski} % replace wit¨h your full name (first name(s) and then family name(s) / surname(s)) including academic degrees
\ctufitauthorsurnames{Golasowski} % replace with your surname(s) / family name(s)
\ctufitauthorgivennames{Ondřej} % replace with your first name(s) / given name(s)
\ctufitsupervisor{Ing.\,Stanislav Jeřábek} % replace with name of your supervisor/advisor (include academic degrees)
\ctufitdepartment{Katedra číslicového návrhu } % replace with the department of your defence
\ctufityear{2023} % replace with the year of your defence
\ctufitdeclarationplace{Praze} % replace with the place where you sign the declaration
\ctufitdeclarationdate{\today} % replace with the date of signature of the declaration
\ctufitabstractCZE{Bakalářská práce se zabývá problematikou emulace v~souvislosti s~výukou principů počítačových architektur a~jejich hardwaru. Na konkrétním příkladě zábavního systému Nintendo Entertainment System ukazuje celý proces vývoje softwarového emulátoru od pochopení základních principů a~nutnou analýzu emulovaného systému přes návrh vhodného řešení na základě zjištěných informací až po samotnou implementaci v~jazyce C++ a~testování výsledku. Snahou implementace je být co nejsrozumitelnější. Součástí řešení je i~univerzální emulační platforma, která je dále využitelná pro jiné projekty mající za cíl vytvořit emulátor počítačového systému. Pro motivaci dalších zájemců o~problematiku je v~poslední kapitole obsažen výčet funkcionalit, které je možné doimplementovat. Byla vytvořena i~podrobná dokumentace, aby se zvýšila přístupnost a~srozumitelnost projektu nejen pro zájemce o~rozšíření platformy, ale i~pro zájemce o~vytvoření vlastního emulátoru s~použitím vytvořené platformy.}
\ctufitabstractENG{The bachelor's thesis is focused on the problematics of software emulation in the context of teaching the principles of computer architectures and associated hardware. There is a~whole emulator development process presented in an example of a~particular computer system, which is the Nintendo Entertainment System. The process consists of understanding the basic principles behind software emulation, analysis of the emulated system, design of the solution based on discovered information, and finally, the implementation of the emulator including testing. The goal of the implementation is to be as comprehensible as possible. The project also includes a universal platform for emulator development. To motivate other students (or hobbyists) interested in the topic, there is a~list of possible extensions of the project in the last chapter of the thesis. Detailed documentation was created to make the project more accessible for emulator developers and potential project contributors.}
\ctufitkeywordsCZE{softwarová emulace, Nintendo Entertainment System, vzdělávání, počítačové architektury, hardware, C++, emulační platforma}
\ctufitkeywordsENG{software emulation, Nintendo Entertainment System, education, computer architectures, hardware, C++, emulation platform}
%%%%%%%%%%%%%%%%%%%%%%%%%%%%%%%%%%
% END FILL IN
%%%%%%%%%%%%%%%%%%%%%%%%%%%%%%%%%%

%%%%%%%%%%%%%%%%%%%%%%%%%%%%%%%%%%
% CUSTOMIZATION of this template
% Skip this part or alter it if you know what you are doing.
%%%%%%%%%%%%%%%%%%%%%%%%%%%%%%%%%%
\RequirePackage{iftex}[2020/03/06]
\iftutex % XeLaTeX and LuaLaTeX
    \RequirePackage{ellipsis}[2020/05/22] %ellipsis workaround for XeLaTeX
\else
    \RequirePackage[utf8]{inputenc}[2018/08/11] %this file encoding
    \RequirePackage{lmodern}[2009/10/30] % vector flavor of Computer Modern font
\fi

% hyperlinks
\RequirePackage[pdfpagelayout=TwoPageRight,colorlinks=false,allcolors=decoration,pdfborder={0 0 0.1}]{hyperref}[2020-05-15]

% uncomment the following to hide all hyperlinks
% \RequirePackage[pdfpagelayout=TwoPageRight,hidelinks]{hyperref}[2020-05-15]

\RequirePackage{pdfpages}[2020/01/28]

\setcounter{secnumdepth}{4} % numbering sections; 4: subsubsection

\definecolor{bfcommon}{RGB}{250, 159, 66}
\definecolor{bfcommonlight}{RGB}{253, 214, 176}
\definecolor{bfaux}{RGB}{71, 188, 255}
\definecolor{bfauxlight}{RGB}{194, 233, 255}

\definecolor{fcaction}{RGB}{253, 214, 176}
\definecolor{fcbranch}{RGB}{194, 233, 255}
\definecolor{fcstart}{RGB}{250, 159, 66}

\definecolor{mastercomponent}{RGB}{71, 188, 255}
\definecolor{slavecomponent}{RGB}{250, 159, 66}
\definecolor{interface}{RGB}{253, 214, 176}

\definecolor{pixel 0}{RGB}{255, 255, 255}
\definecolor{pixel 1}{RGB}{194, 233, 255}
\definecolor{pixel 2}{RGB}{71, 188, 255}
\definecolor{pixel 3}{RGB}{0, 122, 195}

\definecolor{lfsr}{RGB}{0, 122, 195}

\emergencystretch 3em
%%%%%%%%%%%%%%%%%%%%%%%%%%%%%%%%%%
% CUSTOMIZATION of this template END
%%%%%%%%%%%%%%%%%%%%%%%%%%%%%%%%%%


%%%%%%%%%%%%%%%%%%%%%%
% DEMO CONTENTS SETTINGS
% You may choose to modify this part.
%%%%%%%%%%%%%%%%%%%%%%
\usepackage{dirtree}
\usepackage{lipsum,tikz}
\usetikzlibrary{positioning, shapes.geometric, calc, arrows, decorations.pathreplacing, chains}
\usepackage{csquotes}
\usepackage[style=iso-numeric]{biblatex}
\addbibresource{text/bib-database.bib}
%\usepackage{listings} % typesetting of sources
\usepackage{minted} % typesetting of sources
\counterwithin{listing}{chapter}
\usepackage{epigraph}
\usepackage{tikz-timing}
\usetikztiminglibrary[rising arrows]{clockarrows}
\usepackage{xparse}
\usepackage{bytefield}
\usepackage{tabularray}
\usepackage{forloop}
\usepackage{pgfplots}
\usepackage[defaultlines=2,all]{nowidow}

%theorems, definitions, etc.
\theoremstyle{plain}
\newtheorem{theorem}{Věta}
\newtheorem{lemma}[theorem]{Tvrzení}
\newtheorem{corollary}[theorem]{Důsledek}
\newtheorem{proposition}[theorem]{Návrh}
\newtheorem{definition}[theorem]{Definice}
\theoremstyle{definition}
\newtheorem{example}[theorem]{Příklad}
\theoremstyle{remark}
\newtheorem{note}[theorem]{Poznámka}
\newtheorem*{note*}{Poznámka}
\newtheorem{remark}[theorem]{Pozorování}
\newtheorem*{remark*}{Pozorování}
\numberwithin{theorem}{chapter}
%theorems, definitions, etc. END
%%%%%%%%%%%%%%%%%%%%%%
% DEMO CONTENTS SETTINGS END
%%%%%%%%%%%%%%%%%%%%%%

%%%%%%%%%%%%%%%%%%%%%%
% CUSTOM XPARSE COMMANDS
%%%%%%%%%%%%%%%%%%%%%%
% xparse used instead of \newcommand, because is faster and has more features.
% Reference a signal.
%
% Usage:
%
%     \signal[3::0]{C/BE}    ->   C/BE[3::0]
%     \signal*{AD}           ->   AD#
%     \signal*[3::0]{C/BE}   ->   C/BE[3::0]#
%
% By Nathan Typanski from: https://nathantypanski.com/blog/2014-10-29-tikz-timing.html#fn1\textbf
\NewDocumentCommand{\signal}{som}{\texttt{%
		#3%
		\IfValueTF{#2}{[#2]}{}%
		\IfBooleanTF{#1}{\#}{}%
}}


% Draw pixels from a list specified in the first argument with colors defined with
% with prefix specified in the second argument.
% Inspired by an answer by Paul Gaborit at https://tex.stackexchange.com/questions/157080/can-tikz-create-pixel-art-images.
\NewDocumentCommand{\drawpixels}{mm}{
	\foreach \line [count=\y] in #1 {
		\foreach \pix [count=\x] in \line {
			\draw[fill=#2 \pix] (\x,-\y) rectangle +(1,1);
		}
	}
}

%%%%%%%%%%%%%%%%%%%%%%
% XPARSE COMMANDS END
%%%%%%%%%%%%%%%%%%%%%%

\begin{document} 
\frontmatter\frontmatterinit % do not remove these two commands

\includepdf[pages={1-}]{assignment-include.pdf} % replace that file with your thesis assignment provided by study office

\thispagestyle{empty}\cleardoublepage\maketitle % do not remove these three commands

\imprintpage % do not remove this command

\tableofcontents % do not remove this command
%%%%%%%%%%%%%%%%%%%%%%
% list of other contents: figures, tables, code listings, algorithms, etc.
% add/remove commands accordingly
%%%%%%%%%%%%%%%%%%%%%%
\listoffigures % list of figures
\begingroup
\let\clearpage\relax
\listoftables % list of tables
%\lstlistoflistings % list of source code listings generated by the listings package
\listoflistings % list of source code listings generated by the minted package
\endgroup
%%%%%%%%%%%%%%%%%%%%%%
% list of other contents END
%%%%%%%%%%%%%%%%%%%%%%

%%%%%%%%%%%%%%%%%%%
% ACKNOWLEDGMENT
% FILL IN / MODIFY
% This is a place to thank people for helping you. It is common to thank your supervisor.
%%%%%%%%%%%%%%%%%%%
\begin{acknowledgmentpage}
	Děkuji všem hardwarovým vývojářům i~nadšencům za to, co dělají. Od kalkulaček došlo k~velkému posunu a~je mi ctí prezentovat jeden z~hardwarových milníků, který představuje procesor 6502 i~celá konzole Nintendo Entertainment System.\\
	
	Děkuji panu doktorovi Martinovi Novotnému, který mi ukázal, že i~přes velkou sofistikaci současných systémů je stále nejen možné, ale i~důležité se zabývat vývojem počítačů až na úrovni hradel.\\
	
	Velký dík poté patří panu inženýrovi Stanislavovi Jeřábkovi, jenž se ujal vedení práce a~umožnil tak její vznik. Děkuji mu za věcné poznámky k~projektu jako celku~i za připomínky k~textu bakalářské práce.\\
	
    Děkuji komunitě nesdev.org, která i~dnes stále aktualizuje a~přetváří dokumentaci ke konzoli NES, poskytla mi mnoho užitečných faktů, pozorování a testů.\\
    
    A~nakonec, děkuji i~Vám, čtenáři, za čtení tohoto textu a~zájem o~problematiku emulace v~souvislosti nejen se vzděláváním.
\end{acknowledgmentpage} 
%%%%%%%%%%%%%%%%%%%
% ACKNOWLEDGMENT END
%%%%%%%%%%%%%%%%%%%


%%%%%%%%%%%%%%%%%%%
% DECLARATION
% FILL IN / MODIFY
%%%%%%%%%%%%%%%%%%%
% INSTRUCTIONS
% ENG: choose one of approved texts of the declaration. DO NOT CREATE YOUR OWN. Find the approved texts at https://courses.fit.cvut.cz/SFE/download/index.html#_documents (document Declaration for FT in English)
% CZE/SLO: Vyberte jedno z fakultou schvalenych prohlaseni. NEVKLADEJTE VLASTNI TEXT. Schvalena prohlaseni najdete zde: https://courses.fit.cvut.cz/SZZ/dokumenty/index.html#_dokumenty (prohlášení do ZP)
\begin{declarationpage}
Prohlašuji, že jsem předloženou práci vypracoval samostatně a že jsem uvedl veškeré
použité informační zdroje v~souladu s Metodickým pokynem o~dodržování etických
principů při přípravě vysokoškolských závěrečných prací.

Beru na vědomí, že se na moji práci vztahují práva a povinnosti vyplývající ze zákona
č.~121/2000~Sb., autorského zákona, ve znění pozdějších předpisů. V souladu s~ust.
§~2373~odst.~2 zákona č.~89/2012~Sb., občanský zákoník, ve znění pozdějších předpisů,
tímto uděluji nevýhradní oprávnění (licenci) k~užití této mojí práce, a to včetně všech
počítačových programů, jež jsou její součástí či přílohou a veškeré jejich
dokumentace (dále souhrnně jen \uv{Dílo}), a to všem osobám, které si přejí Dílo užít.
Tyto osoby jsou oprávněny Dílo užít jakýmkoli způsobem, který nesnižuje hodnotu
Díla, avšak pouze k~nevýdělečným účelům. Toto oprávnění je časově, teritoriálně
i~množstevně neomezené.
\end{declarationpage}
%%%%%%%%%%%%%%%%%%%
% DECLARATION END
%%%%%%%%%%%%%%%%%%%

\printabstractpage % do not remove this command

%%%%%%%%%%%%%%%%%%%
% ABBREVIATIONS
% FILL IN / MODIFY
% OR REMOVE ENTIRELY
% List the abbreviations in lexicography order.
%%%%%%%%%%%%%%%%%%%
\chapter{Seznam zkratek}
\begin{tabular}{rl}
A 	 & akumulátor \\
ADH &	 address high \\
ADL &	 address low \\
APU &	 Audio Processing Unit\\
ASCII & American Standard Code for Information Interchange \\
ASIC &	 application specific integrated circuit \\
CPU &	 central processing unit\\
DMA & direct memory access \\
DUT & design under test\\
FC &	 Family Computer\\
GUI &	 graphical user interface \\
HBL & horizontal blanking \\
I/O &	 input/output \\
IRQ &	 interrupt request\\
ISA &	 instruction set architecture\\
JSA &	 jazyk symbolických adres\\
LCD & liquid crystal display \\
NES &	 Nintendo Entertainment System\\
NMI &	 non-maskable interrupt\\
NTSC &	 National Television System Committee\\
OAM &	 object attribute memory\\
OOP &	 objektově orientované programování \\
OZ &	 operační znak \\
PAL &	 phase alternating line\\
PC &	 program counter \\
PCM &    pulse code modulation \\
PLA &	 programmable logic array \\
PPU &	 Picture Processing Unit\\
RAM &	 random access memory\\
ROM &	 read only memory\\
S &	 stack pointer \\
VBL & vertical blanking \\
VRAM &	 video random access memory \\
URISC & ultimate reduced instruction set computer \\
USE & Universal System Emulator (platforma 2.0) \\

\end{tabular}
%%%%%%%%%%%%%%%%%%%
% ABBREVIATIONS END
%%%%%%%%%%%%%%%%%%%

\mainmatter\mainmatterinit % do not remove these two commands

%%%%%%%%%%%%%%%%%%%
% THE THESIS
% MODIFY ANYTHING BELOW THIS LINE
%%%%%%%%%%%%%%%%%%%

% Do not forget to include Introduction
%---------------------------------------------------------------
% \chapter{Introduction}
% uncomment the following line to create an unnumbered chapter
\chapter*{Úvod}\addcontentsline{toc}{chapter}{Úvod}\markboth{Úvod}{Úvod}
%---------------------------------------------------------------
\setcounter{page}{1}

Technologický pokrok je nezadržitelný. Nové poznatky umožňují rychlý vývoj sofistikovaného technického vybavení výpočetních číslicových elektronických strojů --- sálových, domácích i~mobilních univerzálních i~specializovaných počítačů, vestavěných řídicích systémů a~dalších zařízení.

Spolu s~technologickým pokrokem přichází mnoho nových informací. Důležitou součástí procesu učení je informace nejen získat, ale i~zpracovat a~porozumět jim, jelikož \enquote{formální osvojování jakýchkoliv faktů bez porozumění se zákonitě promítá do nízké rychlosti i~ekonomičnosti učení a~malé trvalosti paměťové stopy, včetně praktické nevyužitelnosti.}~\cite{Zacharova2012:psychologie}

V~mnoha oblastech však obecné principy zůstávají podobné, ne-li stejné. Přirozeně se tudíž k demonstraci principů nabízí využít jednoduššího systému. Takový systém můžeme získat vytvořením modelu aktuálních složitých systémů, což se prakticky využívá například v~systémech reálného času~\cite{Kubatova2019:src-modely}. Jinou variantou řešení se zabývá tento text; využitím historického systému.

Historické systémy, podobně jako ty moderní, vychází z teoretických matematických konceptů (například programovatelný počítač vycházející z Turingova stroje~\cite{Teuscher2003:turing}). Zároveň jsou poměrně jednoduché, jelikož vznikaly s~technologickými omezeními. Není tedy potřeba vytvářet abstraktní model, ale demonstrovat principy na existujícím systému, což může zvýšit atraktivitu i~užitečnost předávaných informací.

Jedním ze způsobů, jak takový systém přiblížit jakémukoliv zájemci o problematiku, je přenést jej do softwaru, který bude možné spustit na běžně dostupných počítačích. Jednou z možností je takzvaná emulace; výsledný software je nazýván emulátor.

Cílem bakalářské práce je vytvořit emulátor historické herní konzole \emph{Nintendo Entertainment System}. Jelikož je kladem důraz na využití ve výuce, je nutnou součástí návrhu vývoj univerzální platformy, která takový systém zvládne nejen emulovat, ale zároveň zobrazovat informace o~vnitřním stavu systému, jakožto i~umožnit jednoduché modifikace a~přidávání funkcionalit.

Implementační část, hotová emulační platforma, je pouze dílčí výsledek práce. Samotný vývoj emulovaných komponent přináší mnoho zajímavých problémů k~řešení, proto je namístě tento proces důkladně dokumentovat a~vytvořit tak příklad pro uživatele, kteří by chtěli příkladnou implementaci rozšířit, případně na platformě vyvinout emulátor jiného systému. Dílčím cílem práce je tedy seznámit čtenáře s~vývojem a~motivovat jej  ještě důkladněji zkoumat prezentované principy.

Práce je členěna na několik hlavních částí:
\begin{description}
	\item[Představení problematiky] V~této kapitole je představena problematika emulace v~teoretické rovině --- definují se potřebné pojmy. Kapitola popisuje jednak emulaci obecně, jednak konkrétní emulované komponenty.
	\item[Analýza] Analytická, stěžejní kapitola práce, seznámí čtenáře s~uvažovanými variantami řešení. Dochází k~analýze existujících implementací a~k~výběru vhodných technologií a~metodik vzhledem k~emulovaným komponentám.
	\item[Implementace] Implementační kapitola je popisem procesu tvorby emulační platformy a emulovaných komponent.
	\item[Testování] Předposlední kapitola je zaměřena na testování projektu. Popisuje nejen průběžné testování aplikace, ale i~porovnání věrnosti s~reálným systémem. metodou spouštění necertifikovaného kódu na originálním hardwaru.
	\item[Navazující práce] Poslední kapitola je věnována shrnutí zbývajících funkcionalit, které budou implementovány v~dalších verzích emulační platformy. Slouží také jako pobídka dalším uživatelům-vývojářům, kteří by měli zájem projekt rozšířit v~rámci sebevzdělávání.
\end{description}

\begin{note*}[Terminologie]
	Jelikož je bakalářská práce zaměřena na vzdělávací využití, kombinuje odbornost s~populárně-naučným formátem. Počítá se s~faktem, že čtenář se v~oboru informačních technologií již pohybuje, proto se používají anglické termíny tam, kde je to běžné. Nemělo by tedy například čtenáře zaskočit, že se občas jako \emph{programové vybavení} počítače označuje výrazem \emph{software} a~\emph{technické vybavení} počítače jako \emph{hardware}.
\end{note*}

\begin{note*}[Značení]
	V~textu se často používají čísla v šestnáctkové soustavě, jelikož úsporně reprezentují například paměťové adresy. Takové číslice se značí předponou amerického dolaru: \$. Desítková čísla jsou uvedena bez předpony.
	
	Dále se používají různé zkratky; jsou uvedeny v~seznamu zkratek. Neobvyklé zkratky se před jejich použitím v~textu vysvětlují.
\end{note*}

%---------------------------------------------------------------
\chapter{Představení problematiky}
%---------------------------------------------------------------

\epigraph{
	\enquote{We can only see a short distance ahead, but we can see plenty there that needs to be done.}
}{\textit{Computing Machinery and Intelligence}\\ \textsc{Alan Turing}}

\section{Emulace}
V úvodu byl použit pojem emulátor. Pro začátek je tedy vhodné tento pojem oficiálně zavést.

\begin{definition}[Emulátor]
	Emulátor je druh softwaru, který umožňuje běh počítačových programů na jiné platformě, než pro kterou byly původně vytvořeny~\cite{Wikipedia:emulator}.
\end{definition}

\begin{note}[Emulovatelnost]
	Dává smysl se zabývat vytvářením emulátoru, jelikož lze pro každý software vytvořit příslušný emulátor. Lze se odkázat na Churchovu-Turingovu tezi, ze které vyplývá, že ke každému algoritmu existuje ekvivalentní Turingův stroj.
\end{note}

Dle jiné definice pod pojem emulátor spadá i~hardwarové řešení emulátoru. Tímto se však práce nezabývá, proto bude dále brána v~potaz jen již uvedená softwarová emulace.

Emulace se od podobného pojmu, \emph{simulace}, liší především tím, že se na emulátoru spouští originální programové vybavení emulovaného systému. Nedochází tedy k~napodobení funkce, ale celého hardwaru tak, aby byl schopný věrně interpretovat původní program. V~případě této práce se jedná o~interpretaci instrukcí původně obsaženého v paměti ROM.

\begin{example}
V~kontextu herních konzolí lze uvést rozdíl na následujícím příkladu. Simulace by napodobila vzhled a~chování každé jednotlivé hry. Například simulátor příruční herní konzole s~vestavěnou hrou \emph{Tetris} by byla nová implementace hry bez ohledu na hardware, který byl v~konzoli použit. Emulátor by naopak nebral žádný ohled na jakýkoliv software, ale snažil by se věrně napodobit hardwarové vybavení konzole tak, aby bylo možné kopii softwaru (hru) spustit beze změn. Takto je možné provozovat na emulátoru jakékoliv programové vybavení kompatibilní s~daným hardwarem \cite{FulberGarcia2022:simulation-emulation}
\end{example}

\subsection{Způsoby emulace}
Emulaci je možné dělit dle úrovně, na které emulátor pracuje, což je úzce spjato s~teorií počítačových architektur. Na úvod je vhodné se zamyslet nad programováním fyzického počítačového systému, což poskytne přehled o~dostupných zdrojích informací pro vývoj emulátoru. Tato podkapitola tedy odpoví na dvě otázky:
\begin{enumerate}
	\item V~jaké formě bude spouštěný software?
	\item Na jaké úrovni se tento software zpracuje?
\end{enumerate}

Běžné počítačové systémy odpovídají teoretickému modelu programovatelného počítače.

\begin{definition}[Programovatelný počítač]
	Programovatelný počítač je takový počítač, který čte instrukce z~elektronické paměti, kde jsou uloženy.~\cite{Wikipedia:programovatelny-pocitac}
\end{definition}

Komponentou počítače, která je zodpovědná za řízení, je většinou \emph{procesor}. Proto má smysl se nejdříve zamýšlet nad úrovní abstrakce procesoru, jelikož je to právě ta komponenta, která bude zpracovávat programy a~řídit komponenty ostatní.

Instrukce bývají v paměti číslicových počítačů reprezentovány jako strojový kód, který většinou vzniká překladem z~jazyka vyšší abstrakce (například JSA)~\cite{Kubatova2018:SAP}, což ilustruje diagram~\ref{fig:abstrakce-sw}. Jelikož je strojový kód nativní způsob zpracování instrukcí a~zároveň se v~této formě běžně distribuuje software, emulátor by měl pracovat právě s~touto reprezentací. Tím se získala odpověď na první otázku.

\begin{figure}[ht!]
	\centering
	\caption{~Úrovně abstrakce softwaru}\label{fig:abstrakce-sw}
	\begin{tikzpicture}[node distance=2cm] 
		\tikzstyle{uroven} = [rectangle, rounded corners, minimum width=5cm, minimum height=1cm,text centered, draw=black]
		\tikzstyle{arrow} = [thick,->,>=stealth]
		
		\node (vyssijazyk) [uroven] {Vyšší programovací jazyk};
		\node (jsa) [uroven, below of=vyssijazyk] {Jazyk symbolických instrukcí};
		\node (strojkod) [uroven, below of=jsa, fill=headbackgroundgray] {Strojový kód};
		\node (signaly) [uroven, below of=strojkod] {Řídící signály};
		
		\draw [arrow] (vyssijazyk) -- (jsa);
		\draw [arrow] (jsa) -- (strojkod);
		\draw [arrow] (strojkod) -- (signaly);
	\end{tikzpicture}
\end{figure}

Druhá otázka se již zabývá přiřazení smyslu jednotlivým instrukcím. Množina podporovaných instrukcí včetně dalších potřebných informací (především o způsobu reprezentace a ukládání dat) je součástí architektury procesoru (ISA)~\cite{Kubatova2018:SAP}.

Z výše uvedeného vyplývá, že pro zpracování instrukcí tak, jako to dělal původní hardware, stačí jen přesně napodobit chování jednotlivých instrukcí dle popisu architektury, bez dalšího zamýšlení se, jak je procesor konkrétně implementován. To představuje nejvyšší úroveň abstrakce. 

Některé programy se však občas spoléhají na nedokumentované chování procesorů, kde je již nutné pracovat na nižší úrovni. Dle prof. Kubátové se jedná o úroveň předávání dat mezi registry, na které pracuje i~emulátor bakalářské práce.

Existují ještě dvě nižší úrovně, úroveň logických hradel a~úroveň tranzistorů~\cite{Kubatova2018:SAP}. Tyto dvě úrovně již však vyžadují znalost konkrétní hardwarové implementace, která bývá obchodním tajemstvím. Výhodou je, že ve své podstatě nevyžaduje vůbec znalost o~funkci procesoru jako takovém a~zároveň nejvěrněji implementuje jeho funkčnost. Velkými nevýhodami jsou obtížnost, častá absence potřebných informací a~při implementaci v softwaru i~velká náročnost na prostředky, jelikož se emuluje každý řídicí signál, tedy nejnižší úroveň řízení dle diagramu~\ref{fig:abstrakce-sw}. Zájemce o~tuto úroveň lze odkázat na projekt Visual6502~\cite{Visual6502:slides}.

Všechny úrovně shrnuje diagram~\ref{fig:abstrakce-hw}, kde je zvýrazněna úroveň používaná v této práci.

\begin{figure}[ht!]
	\centering
	\caption{~Úrovně abstrakce hardwaru}\label{fig:abstrakce-hw}
	\begin{tikzpicture}[node distance=2cm] 
		\tikzstyle{uroven} = [rectangle, rounded corners, minimum width=5cm, minimum height=1cm,text centered, draw=black]
		\tikzstyle{arrow} = [thick,->,>=stealth]
		
		\node (cpu) [uroven] {Procesor};
		\node (reg) [uroven, below of=cpu, fill=headbackgroundgray] {Registr};
		\node (hradlo) [uroven, below of=reg] {Hradlo};
		\node (tranzistor) [uroven, below of=hradlo] {Tranzistor};
	
		\draw [arrow] (cpu) -- (reg);
		\draw [arrow] (reg) -- (hradlo);
		\draw [arrow] (hradlo) -- (tranzistor);
	\end{tikzpicture} 
\end{figure}

\section{Nintendo Entertainment System}
Konzole Nintendo Entertainment System, často zkracována jako NES, je osmibitový zábavní počítačový systém firmy Nintendo, který byl vydán nejprve v Japonsku jako Family Computer (FC, \enquote{Famicom}). V~České republice je tento systém znám především díky mnoha klonům, které byly levnější a~dostupnější, než oficiální systém. Tyto klony nejprve používaly kopie původního hardwaru, poté se objevily hardwarové emulátory založené na ASIC, které celou konzoli zmenšili do jednoho čipu (proto přezdívány NES-on-a-chip). Zájemce o~podrobnou historii NES lze odkázat na Wikipedii~\cite{Wikipedia:NES}~\cite{Wikipedia:famiclone} a článek~\cite{Svara:polystation}.

Systém se skládá z několika hlavních komponent, které spolu komunikují pomocí dvou sběrnic. Nejprve budou představeny komponenty na hlavní sběrnici, což je ilustrováno na obrázku~\ref{fig:nes-hlavnisbernice}. Druhá sběrnice je pak ilustrována na TODO.

\subsection{Hlavní sběrnice}
Hlavní sběrnice je adresována 16~bity a~přenáší 8~datových bitů. Komunikaci na hlavní sběrnici řídí klon procesoru \emph{6502}, do jeho adresního prostoru jsou mapovány komponenty dle tabulky~\ref{tab:cpu-adresniprostor}.

\begin{table}[ht!]
	\centering
	\caption{~Adresní prostor CPU}\label{tab:cpu-adresniprostor}
	\begin{tabular}{|c|c|}
		\hline
		Adresní rozsah & Zařízení \\
		\hline
		\$0000–\$07FF & RAM \\
		\hline
		\$0800–\$1FFF & Zrcadlo \$0000–\$07FF \\
		\hline
		\$2000–\$2007 & Registry PPU \\
		\hline
		\$2008–\$3FFF & Zrcadlo \$2000–\$2007 \\
		\hline
		\$4000–\$4017 & Registry APU a I/O \\
		\hline
		\$4018–\$401F & Nepoužíváno \\
		\hline
		\$4020–\$FFFF & Game Pak \\
		\hline
	\end{tabular}
\end{table}

Komunikaci na hlavní sběrnici řídí klon procesoru \emph{6502} vyrobený firmou Ricoh. Přímo v~procesoru se nachází čip pro generování zvuku, přezdívaný jako Audio Processing Unit (APU). Tento čip komunikuje také na hlavní sběrnici.  Procesor má k~dispozici 2~kB paměti RAM. Na sběrnici se dále nachází grafický čip \emph{2C02}, označovaný jako Picture Processing Unit (PPU). Jako poslední je na sběrnici několik vstupně-výstupních (I/O) rozhraní: slot pro médium typu cartridge, obchodně označovaná jako Game~Pak, pomocí níž se distribuoval veškerý software pro konzoli a~porty pro periferie, především herní ovladače.



\subsection{Grafická sběrnice}
Systém obsahuje i~vedlejší sběrnici, kterou ovládá PPU. Tato sběrnice obsahuje další paměť RAM o~kapacitě 2~kB. Do paměťového prostoru

\begin{figure}[ht!]
	\centering
	\caption{~Hlavní sběrnice konzole NES}\label{fig:nes-hlavnisbernice}
	% Kód pro renderování sběrnice byl inspirován příspěvkem od uživatele Ignasi na StackExchange.
	% https://tex.stackexchange.com/questions/319864/how-to-improve-my-draw-for-i2c-bus
	\begin{tikzpicture}[
		master/.style={draw, rounded corners, fill=mastercomponent, minimum height=15mm, minimum width=2.5cm},
		slave/.style={draw, rounded corners, fill=slavecomponent, minimum height=10mm, minimum width=2.5cm},
		slot/.style={draw, rounded corners, fill=interface, minimum height=10mm, minimum width=2.5cm}
		]
		
		\node[master] (m) {CPU (6502)};
		\node[slave, below right=3mm and 7mm of m] (s1) {RAM};
		\node[slave, right= 6mm of s1] (s2) {PPU (2C02)};
		\node[slave, right= 6mm of s2] (s3) {APU (2A03)};
		\node[slot, right= 6mm of s3] (s4) {I/O};
		
		\draw[thick] (m)--(m-|s4.east) node[above]{Hlavní sběrnice};
		\foreach \i in {1,2,3,4}{
			\draw[fill=black] (s\i)--(s\i|-m) circle (2pt);
		}
	
		\node[slave, below left=of s4] (gamepak) {Game Pak};	
		\draw[dashed] (s4)--(gamepak);


		\node[slave, below=of s4] (periferie) {Periferie};	
		\draw[dashed] (s4)--(periferie);
	\end{tikzpicture}
\end{figure}
	
\subsection{CPU 6502}
Základem NES je kopie procesoru \emph{6502}, označená jako \emph{2A03}. Aby se zabránilo porušení patentu, v~hardwaru se přerušilo několik cest tak, aby se učinil desítkový režim (viz dále) procesoru nefunkční. Tento procesor byl navíc doplněn o~zvukový syntezátor, přezdívaný jako Audio Processing Unit (APU). Aby byla zachována univerzálnost řešení, je v~práci implementován původní procesor 6502 a~funkcionality specifické pro 2A03 jsou řešeny zvlášť, proto jsou komponenty takto popisovány i~v~této části.


\subsubsection{Specifika Ricoh 2A03}

\subsection{PPU}
Nedílnou součástí herní konzole je grafický výstup, tu zajišťuje čip 2C02, přezdívaný jako Picture Processing Unit (PPU). Existují dvě varianty specifické pro konkrétní trhy v~závislosti na typu výstupního signálu --- NTSC a PAL. Tento čip během zpracování grafických dat generuje přímo analogový signál, tudíž každý hodinový takt odpovídá jednomu obrazovému bodu.






%---------------------------------------------------------------
\chapter{Analýza}
%---------------------------------------------------------------

%---------------------------------------------------------------
\chapter{Implementace}
%--------------------------------------------------------------- % include `text.tex' from `text/' subdirectory

\appendix\appendixinit % do not remove these two commands

\chapter{Nějaká příloha}


Sem přijde to, co nepatří do hlavní části.

\begin{figure}[p!]
	\centering
	\caption{Přehledové hardwarové schéma konzole NES. Obrázek \enquote{NES-001 Console} vytvořil schenkzoola pod licencí CC~BY~4.0.}
	\label{fig:nes001-hw}
	\includegraphics[width=0.95\textheight, angle=270]{images/NES-001.pdf}
\end{figure} % include `appendix.tex' from `text/' subdirectory

\backmatter % do not remove this command

\printbibliography[heading=bibintoc, title={Bibliografie}] % print out the BibLaTeX-generated bibliography list

\chapter{Obsah přiloženého média}


	\dirtree{%
		.1 readme.txt\DTcomment{stručný popis obsahu média}.
		.1 implementace.
		.2 use-main.zip\DTcomment{kopie repozitáře s~implementací projektu}.
		.1 text.
		.2 bp-main.zip\DTcomment{kopie repozitáře se sazbou textu ve formátu~\LaTeX{}}.
		.2 thesis.pdf\DTcomment{text práce ve formátu PDF}.
	}
 % include `medium.tex' from `text/' subdirectory

\end{document}
