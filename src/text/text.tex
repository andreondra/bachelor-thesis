% Do not forget to include Introduction
%---------------------------------------------------------------
% \chapter{Introduction}
% uncomment the following line to create an unnumbered chapter
\chapter*{Úvod}\addcontentsline{toc}{chapter}{Úvod}\markboth{Úvod}{Úvod}
%---------------------------------------------------------------
\setcounter{page}{1}

Technologický pokrok je nezadržitelný. Nové poznatky umožňují rychlý vývoj sofistikovaného technického vybavení výpočetních číslicových elektronických strojů --- sálových, domácích i~mobilních univerzálních i~specializovaných počítačů, vestavěných řídicích systémů a~dalších zařízení.

Spolu s~technologickým pokrokem přichází mnoho nových informací. Důležitou součástí procesu učení je informace nejen získat, ale i~zpracovat a~porozumět jim, jelikož \enquote{formální osvojování jakýchkoliv faktů bez porozumění se zákonitě promítá do nízké rychlosti i~ekonomičnosti učení a~malé trvalosti paměťové stopy, včetně praktické nevyužitelnosti.}~\cite{Zacharova2012:psychologie}

V~mnoha oblastech však obecné principy zůstávají podobné, ne-li stejné. Přirozeně se tudíž k demonstraci principů nabízí využít jednoduššího systému. Takový systém můžeme získat vytvořením modelu aktuálních složitých systémů, což se prakticky využívá například v~systémech reálného času~\cite{Kubatova2019:src-modely}. Jinou variantou řešení se zabývá tento text; využitím historického systému.

Historické systémy, podobně jako ty moderní, vychází z teoretických matematických konceptů (například programovatelný počítač vycházející z Turingova stroje~\cite{Teuscher2003:turing}). Zároveň jsou poměrně jednoduché, jelikož vznikaly s~technologickými omezeními. Není tedy potřeba vytvářet abstraktní model, ale demonstrovat principy na existujícím systému, což může zvýšit atraktivitu i~užitečnost předávaných informací.

Jedním ze způsobů, jak takový systém přiblížit jakémukoliv zájemci o problematiku, je přenést jej do softwaru, který bude možné spustit na běžně dostupných počítačích. Jednou z možností je takzvaná emulace; výsledný software je nazýván emulátor.

Cílem bakalářské práce je vytvořit emulátor historické herní konzole \emph{Nintendo Entertainment System}. Jelikož je kladem důraz na využití ve výuce, je nutnou součástí návrhu vývoj univerzální platformy, která takový systém zvládne nejen emulovat, ale zároveň zobrazovat informace o~vnitřním stavu systému, jakožto i~umožnit jednoduché modifikace a~přidávání funkcionalit.

Implementační část, hotová emulační platforma, je pouze dílčí výsledek práce. Samotný vývoj emulovaných komponent přináší mnoho zajímavých problémů k~řešení, proto je namístě tento proces důkladně dokumentovat a~vytvořit tak příklad pro uživatele, kteří by chtěli příkladnou implementaci rozšířit, případně na platformě vyvinout emulátor jiného systému. Dílčím cílem práce je tedy seznámit čtenáře s~vývojem a~motivovat jej  ještě důkladněji zkoumat prezentované principy.

Práce je členěna na několik hlavních částí:
\begin{description}
	\item[Představení problematiky] V~této kapitole je představena problematika emulace v~teoretické rovině --- definují se potřebné pojmy. Kapitola popisuje jednak emulaci obecně, jednak konkrétní emulované komponenty.
	\item[Analýza] Analytická, stěžejní kapitola práce, seznámí čtenáře s~uvažovanými variantami řešení. Dochází k~analýze existujících implementací a~k~výběru vhodných technologií a~metodik vzhledem k~emulovaným komponentám.
	\item[Implementace] Implementační kapitola je popisem procesu tvorby emulační platformy a emulovaných komponent.
	\item[Testování] Předposlední kapitola je zaměřena na testování projektu. Popisuje nejen průběžné testování aplikace, ale i~porovnání věrnosti s~reálným systémem. metodou spouštění necertifikovaného kódu na originálním hardwaru.
	\item[Navazující práce] Poslední kapitola je věnována shrnutí zbývajících funkcionalit, které budou implementovány v~dalších verzích emulační platformy. Slouží také jako pobídka dalším uživatelům-vývojářům, kteří by měli zájem projekt rozšířit v~rámci sebevzdělávání.
\end{description}

%---------------------------------------------------------------
\chapter{Představení problematiky}
%---------------------------------------------------------------

\epigraph{
	\enquote{We can only see a short distance ahead, but we can see plenty there that needs to be done.}
}{\textit{Computing Machinery and Intelligence}\\ \textsc{Alan Turing}}

\section{Emulace}
V úvodu byl použit pojem emulátor. Pro začátek je tedy vhodné tento pojem oficiálně zavést.

\begin{definition}[Emulátor]
	Emulátor je druh softwaru, který umožňuje běh počítačových programů na jiné platformě, než pro kterou byly původně vytvořeny~\cite{Wikipedia:emulator}.
\end{definition}

\begin{note}[Emulovatelnost]
	Dává smysl se zabývat vytvářením emulátoru, jelikož lze pro každý software vytvořit příslušný emulátor. Lze se odkázat na Churchovu-Turingovu tezi, ze které vyplývá, že ke každému algoritmu existuje ekvivalentní Turingův stroj.
\end{note}

Dle jiné definice pod pojem emulátor spadá i~hardwarové řešení emulátoru. Tímto se však práce nezabývá, proto bude dále brána v~potaz jen již uvedená softwarová emulace.

Emulace se od podobného pojmu, \emph{simulace}, liší především tím, že se na emulátoru spouští originální programové vybavení emulovaného systému. Nedochází tedy k~napodobení funkce, ale celého hardwaru tak, aby byl schopný věrně interpretovat původní program. V~případě této práce se jedná o~interpretaci instrukcí původně obsaženého v paměti ROM.

\begin{example}
V~kontextu herních konzolí lze uvést rozdíl na následujícím příkladu. Simulace by napodobila vzhled a~chování každé jednotlivé hry. Například simulátor příruční herní konzole s~vestavěnou hrou \emph{Tetris} by byla nová implementace hry bez ohledu na hardware, který byl v~konzoli použit. Emulátor by naopak nebral žádný ohled na jakýkoliv software, ale snažil by se věrně napodobit hardwarové vybavení konzole tak, aby bylo možné kopii softwaru (hru) spustit beze změn. Takto je možné provozovat na emulátoru jakékoliv programové vybavení kompatibilní s~daným hardwarem \cite{FulberGarcia2022:simulation-emulation}
\end{example}

\subsection{Způsoby emulace}
Emulaci je možné dělit dle úrovně, na které emulátor pracuje, což je úzce spjato s~teorií počítačových architektur. Na úvod je vhodné se zamyslet nad programováním fyzického počítačového systému.

Běžné počítačové systémy odpovídají teoretickému modelu programovatelného počítače.

\begin{definition}[Programovatelný počítač]
	Programovatelný počítač je takový počítač, který čte instrukce z~elektronické paměti, kde jsou uloženy.
\end{definition}


TODO: vyjít z programovatelného počítače, popsat, jak se programuje pro fyzický systém instrukcemi dle ISA

Emulaci je možné dělit dle úrovně, na které emulátor pracuje. Úrovní je myšlen stupeň abstrakce hardwaru. Obecně dle prof.\,Kubátové~\cite{Kubatova2018:SAP} rozlišujeme čtyři základní úrovně, které jsou hierarchicky uspořádány, jak uvádí obrázek \ref{fig:abstrakce-rizeni}. Nabízí se příměr k~abstrakci v~programování --- tak, jak je abstraktní datový typ vektor rozhraním pro primitivnější datové typy, poskytuje procesor rozhraní nad svou vnitřní implementací. 

\begin{figure}[ht!]
	\centering
	\caption{~Úrovně abstrakce hardwaru}\label{fig:abstrakce-rizeni}
	\begin{tikzpicture}[node distance=2cm] 
		\tikzstyle{uroven} = [rectangle, rounded corners, minimum width=5cm, minimum height=1cm,text centered, draw=black]
		\tikzstyle{arrow} = [thick,->,>=stealth]
		
		\node (cpu) [uroven] {Procesor};
		\node (reg) [uroven, below of=cpu, fill=headbackgroundgray] {Registr};
		\node (hradlo) [uroven, below of=reg] {Hradlo};
		\node (tranzistor) [uroven, below of=hradlo] {Tranzistor};
	
		\draw [arrow] (cpu) -- (reg);
		\draw [arrow] (reg) -- (hradlo);
		\draw [arrow] (hradlo) -- (tranzistor);
	\end{tikzpicture} 
\end{figure}

\begin{description}
\item[Kapitola 1] Lorem ipsum dolor sit amet, consectetuer adipiscing elit. Curabitur sagittis hendrerit ante. Class aptent taciti sociosqu ad litora torquent per conubia nostra, per inceptos hymenaeos. Cras pede libero, dapibus nec, pretium sit amet, tempor quis.

\item[Kapitola 2] Lorem ipsum dolor sit amet, consectetuer adipiscing elit. Curabitur sagittis hendrerit ante. Class aptent taciti sociosqu ad litora torquent per conubia nostra, per inceptos hymenaeos. Cras pede libero, dapibus nec, pretium sit amet, tempor quis.

\item[Kapitola 3] Lorem ipsum dolor sit amet, consectetuer adipiscing elit. Curabitur sagittis hendrerit ante. Class aptent taciti sociosqu ad litora torquent per conubia nostra, per inceptos hymenaeos. Cras pede libero, dapibus nec, pretium sit amet, tempor quis.

\item[Kapitola 4] Lorem ipsum dolor sit amet, consectetuer adipiscing elit. Curabitur sagittis hendrerit ante. Class aptent taciti sociosqu ad litora torquent per conubia nostra, per inceptos hymenaeos. Cras pede libero, dapibus nec, pretium sit amet, tempor quis.
\end{description}

\lipsum[2]

\section{Lorem ipsum dolor sit amet}

\lipsum[3-5]
